\chapter {Hello World!}
\lstset{language=Java}

\section {Wozu „Hello World!“?}

Fast alle Bücher, Tutorials, Vorkurshefte und sonstige Programmierbeispiele beginnen mit einem einfachen Programm, das nichts tut als „Hello World!“ auf der Konsole auszugeben.

Im Grund behandelt man mit dem „Hello World!“-Programm zwei Fragen:
\begin{itemize}
\item Was und an welchen Stellen kann ich im Quelltext ändern, um ein tolleres Programm zu erhalten?
\item Wie komme ich von der Quelltext-Datei zum laufenden Programm?
\end{itemize}

\subsection{Bestandteile des Hello-World-Progammes}
In den meisten Programmiersprachen gibt es eine Art „Rahmen“ der die eigentliche Programmlogik umgibt.
In Java sieht das Hello-World-Programm wie folgt aus: 
\begin{lstlisting}
/*
 * Vorkurs fuer Informatik
 * Hello.java
 *
 */
public class Hello {
  public static void main(String args[]) {
    System.out.println("Hello World!");	// gibt 'Hello World' aus.
  }
}
\end{lstlisting}

Die erste Zeile enthält unter anderem den Namen des Programms nämlich „Hello“. Die Datei mit dem Quelltext muss demzufolge \lstinline$Hello.java$ heißen. \lstinline$public class$ bedeutet, dass wir eine nach außen sichtbare Klasse definieren. Warum man das so machen muss ist ersteinmal egal.\\
\begin{lstlisting}
public static void main(String args[]) {
\end{lstlisting}
Markiert den Anfang der Main-Funktion des Programmes. In der Main-Funktion stehen die Befehle, die nach Programmstart als erstes abgearbeitet werden. \\
\begin{lstlisting}
System.out.println("Hello World!");
\end{lstlisting}
Das ist der Befehl der „Hello World“ ausgibt. \lstinline$System.out.println$ ist der Befehl der Text ausgibt, „Hello World!“ der Text den wir ausgeben wollen. \\
Die beiden schließenden (geschweiften) Klammern sind das Ende der Main-Funktion und das Ende des Programmes. 
Wenn du ein anderes Programm schreiben willst, musst du lediglich die „\lstinline$System.out.println()$“-Zeile durch andere Befehle ersetzen.
Damit bleiben allerdings noch einige Stellen des Programmcodes unerklärt: \lstinline$//$ leitet einen Einzeiligen Kommentar ein. Der Rest der Zeile, beginnend mit den beiden Schrägstrichen wird beim Übersetzen des Programmes ignoriert.
Mehrzeilige Kommentare beginnen mit \lstinline$/*$ und enden mit \lstinline$*/$. Der Übersicht halber führt man oftmals die Reihe mit Sternen von unten bis oben fort.


\subsection{Das Programm ausführen}
Um das Programm auszuführen, musst du zuerst den Quelltext des Programms in Bytecode umwandeln lassen und dann die Java-Virtual-Maschine aufrufen, die den Bytecode aufsführt. 
Für das Übersetzen des Quelltextes in Bytecode ist der Java-Compiler zuständig
Unter UNIX ruft man den Java-Compiler aus dem Terminal mit „javac Hello.java“ auf. Der Compiler hat nun, sofern das Programm keine Fehler enthält eine Datei mit dem Namen „Hello.class“ erzeugt (oder eine ältere Version überschrieben) – diese Datei enthält den Bytecode.
Die JVM kannst du mit dem Befehl „java“ aufrufen. „java Hello“ führt den Bytecode in Hello.class aus.

\section {Aufgaben}
\begin{enumerate}
\item Sieh dir den Quelltext in Hello.java an
\item Kompiliere das Programm und führe es aus
\item Ändere den ausgegebenen Text
\item Ändere den Namen des Programms
\item Führe das geänderte Progamm aus
\end{enumerate}

Normalerweise bearbeitet man Quelltext nicht in einem einfachen Texteditor sondern einer weit praktischeren Entwicklungsumgebung.
Für Java bietet sich als Entwicklungsumgebung das Programm \lstinline$eclipse$ an.
\begin{itemize}
\item Starte Eclipse und erstelle ein neues Projekt für das Hello-World-Programm.
\item Füge eine Quelltextdatei mit dem Hello-World-Programm zum Projekt hinzu und führe es aus.
\end{itemize}
